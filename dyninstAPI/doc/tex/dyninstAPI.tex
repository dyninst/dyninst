\documentclass[twopages,a4paper]{article}
\usepackage[left=90pt,right=90pt]{geometry}
\usepackage{graphicx}
\usepackage{imakeidx}
\usepackage[nottoc,numbib]{tocbibind}
\usepackage{xspace}
\usepackage{courier}
\usepackage{listings}
\makeindex[name=terms,title=Index of terms,columns=1]

\title{Dyninst Programmer's Guide}
\begin{document}


\maketitle

\newlength\tindent
\setlength{\tindent}{\parindent}
\setlength{\parindent}{0pt}
\renewcommand{\indent}{\hspace*{\tindent}}


\includegraphics{logo}

\pagebreak
\tableofcontents

\pagebreak

\section{Introduction}
\index[terms]{abc}\index[terms]{abc}\index[terms]{abc}

The normal cycle of developing a program is to edit the source code, compile it, and then execute the resulting binary.  However, sometimes this cycle can be too restrictive.  We may wish to change the program while it is executing or after it has been linked, thus avoiding the process of  re-compiling, re-linking, or even re-executing the program to change the binary.  At first, this may seem like a bizarre goal, however, there are several practical reasons why we may wish to have such a system.  For example, if we are measuring the performance of a program and discover a performance problem, it might be necessary to insert additional instrumentation into the program to understand the problem.  Another application is performance steering; for large simulations, computational scientists often find it advantageous to be able to make modifications to the code and data while the simulation is executing.\\


This document describes an Application Program Interface (API) to permit the insertion of code into a computer application that is either running or on disk.  The API for inserting code into a running application, called dynamic instrumentation, shares much of the same structure as the API for inserting code into an executable file or library, known as static instrumentation.  The API also permits changing or removing subroutine calls from the application program.  Binary code changes are useful to support a variety of applications including debugging, performance monitoring, and to support composing applications out of existing packages.  The goal of this API is to provide a ma-chine independent interface to permit the creation of tools and applications that use runtime and static code patching.  The API and a simple test application are described in \cite{api-code-patching}.  This API is based on the idea of dynamic instrumentation described in [3].

The key features of this interface are the abilities to:
\begin{itemize}
	\item 	Insert and change instrumentation in a running program.
	\item Insert instrumentation into a binary on disk and write a new copy of that binary back to disk.
	\item Perform static and dynamic analysis on binaries and processes.
\end{itemize}


The goal of this API is to keep the interface small and easy to understand.  At the same time, it needs to be sufficiently expressive to be useful for a variety of applications.  We accomplished this goal by providing a simple set of abstractions and a way to specify which code to insert into the application . \footnote{To generate more complex code, extra (initially un-called) subroutines can be linked into the application program, and calls to these subroutines can be inserted at runtime via this interface.
}

\pagebreak
\newcommand{\mutator}[0]{\textit{mutator}\xspace}
\newcommand{\mutatorprocess}[0]{\textit{mutator process}\xspace}
\newcommand{\mutatee}[0]{\textit{mutatees}\xspace}
\newcommand{\mutatees}[0]{\textit{mutatees}\xspace}
\newcommand{\Mutatees}[0]{\textit{Mutatees}\xspace}
\newcommand{\point}[0]{\textit{point}\xspace}
\newcommand{\points}[0]{\textit{points}\xspace}

\newcommand{\snippet}[0]{\textit{snippet}\xspace}

\newcommand{\snippets}[0]{\textit{snippets}\xspace}
\newcommand{\Snippets}[0]{\textit{snippets}\xspace}

\newcommand{\addressspace}[0]{\textit{address space}\xspace}

\newcommand{\process}[0]{\textit{process}\xspace}

\newcommand{\binary}[0]{\textit{binary}\xspace}

\newcommand{\function}[0]{\textit{function}\xspace}

\newcommand{\Functions}[0]{\textit{Functions}\xspace}
\newcommand{\functions}[0]{\textit{functions}\xspace}

\newcommand{\variable}[0]{\textit{variable}\xspace}
\newcommand{\variables}[0]{\textit{variable}\xspace}

\newcommand{\controlflowgraph}[0]{\textit{control flow graph}\xspace}
\newcommand{\basicblocks}[0]{\textit{baisc blocks}\xspace}
\newcommand{\edges}[0]{\textit{edges}\xspace}
\newcommand{\loops}[0]{\textit{loops}\xspace}
\newcommand{\instructions}[0]{\textit{instructions}\xspace}
\newcommand{\types}[0]{\textit{types}\xspace}
\newcommand{\localvariables}[0]{\textit{local variables}\xspace}

\newcommand{\functionparameters}[0]{\textit{function parameters}\xspace}
\newcommand{\sourcecodelineinformation}[0]{\textit{source code line inforamtion}\xspace}
\newcommand{\image}[0]{\textit{image}\xspace}

\newcommand{\entrypoints}{\textit{entry points}\xspace}



\section{Abstractions}
The DyninstAPI library provides an interface for instrumenting and working with binaries and processes.  The user writes a \mutator, which uses the DyninstAPI library to operate on the application.  The process that contains the \mutator and DyninstAPI library is known as the \mutatorprocess.  The \mutatorprocess  operates on other processes or on-disk binaries, which are known as \mutatees.\\\\   
The API is based on abstractions of a program.  For dynamic instrumentation, it can be based on the state while in execution.  The two primary abstractions in the API are \points and \snippets.  A \point is a location in a program where instrumentation can be inserted.  A \snippet is a representation of some executable code to be inserted into a program at a point.  For example, if we wished to record the number of times a procedure was invoked, the \point would be entry point of the procedure, and the \snippets would be a statement to increment a counter.  \Snippets can include conditionals and function calls.\\\\

\Mutatees are represented using an \addressspace abstraction.  For dynamic instrumentation, the \addressspace represents a process and includes any dynamic libraries loaded with the process. For static instrumentation, the \addressspace includes a disk executable and includes any dynamic library files on which the executable depends.  The \addressspace abstraction is extended by \process and \binary abstractions for dynamic and static instrumentation.  The \process abstraction represents information about a running process such as threads or stack state.  The \binary abstraction represents information about a binary found on disk.\\

The code and data represented by an \addressspace is broken up into \function and \variable abstractions.  \Functions contain \points, which specify locations to insert instrumentation.  \Functions also contain a \controlflowgraph abstraction, which contains information about basic blocks, edges, loops, and instructions.  If the \mutatee contains debug information, DyninstAPI will also provide abstractions about \variable and \function \types, \localvariables, \functionparameters, and \sourcecodelineinformation.  The collection of \functions and \variables in a \mutatee is represented as an \image.\\\\

The API includes a simple type system based on structural equivalence.  If mutatee programs have been compiled with debugging symbols and the symbols are in a format that Dyninst understands, type checking is performed on code to be inserted into the mutatee.  See Section~\ref{sec:type_system} for a complete description of the type system.\\

Due to language constructs or compiler optimizations, it may be possible for multiple functions to \textit{overlap} (that is, share part of the same function body) or for a single function to have multiple \entrypoints .  In practice, it is impossible to determine the difference between multiple overlapping functions and a single function with multiple entry points.  The DyninstAPI uses a model where each function (BPatch\_function object) has a single entry point, and multiple functions may overlap (share code).  We guarantee that instrumentation inserted in a particular function is only executed in the context of that function, even if instrumentation is inserted into a location that exists in multiple functions. \index[terms]{abc}
\pagebreak
\include{examples}


\section{Interface}
\subsection{CLASS BPATCH}
\subsection{CALLBACKS}
\subsubsection{Asynchronous Callbacks}
\subsubsection{Code Discovery Callbacks}
\subsubsection{Code Overwrite Callbacks}
\subsubsection{Dynamic Calls}
\subsubsection{Dynamic Libraries}
\subsubsection{Errors}
\subsubsection{Exec}
\subsubsection{Exit}
\subsubsection{Fork}
\subsubsection{One Time Code}
\subsubsection{Signal Handler}
\subsubsection{Stopped Threads}
\subsubsection{User-triggered callbacks}
\subsection{CLASS BPATCH\_ADDRESSSPACE}
\subsection{CLASS BPATCH\_PROCESS}
\subsection{CLASS BPATCH\_THREAD}
\subsection{CLASS BPATCH\_BINARYEDIT}
\subsection{CLASS BPATCH\_SOURCEOBJ}
\subsection{CLASS BPATCH\_FUNCTION}
\subsection{CLASS BPATCH\_POINT}
\subsection{CLASS BPATCH\_IMAGE}
\subsection{CLASS BPATCH\_OBJECT}
\subsection{CLASS BPATCH\_MODULE}
\subsection{CLASS BPATCH\_SNIPPET}
\subsection{CLASS BPATCH\_TYPE}
\subsection{CLASS BPATCH\_VARIABLEEXPR}
\subsection{CLASS BPATCH\_FLOWGRAPH}
\subsection{CLASS BPATCH\_BASICBLOCK}
\subsection{CLASS BPATCH\_EDGE}
\subsection{CLASS BPATCH\_BASICBLOCKLOOP}
\subsection{CLASS BPATCH\_LOOPTREENODE}
\subsection{CLASS BPATCH\_REGISTER}
\subsection{CLASS BPATCH\_SOURCEBLOCK}
\subsection{CLASS BPATCH\_CBLOCK}
\subsection{CLASS BPATCH\_FRAME}
\subsection{CLASS STACKMOD}
\subsection{CONTAINER CLASSES}
\subsubsection{Class std::vecotr}
\subsubsection{Class BPatch\_Set}
\subsection{MEMORY ACCESS CLASSES}
\subsubsection{Class BPatch\_memoryAccess}
\subsubsection{Class BPatch\_addrSpec\_NP}
\subsubsection{Class BPatch\_countSpec\_NP}
\subsection{TYPE SYSTEM}\label{sec:type_system}

\pagebreak
\section{Using Dyninst API with the component libraries}
\section{Using the API}
\subsection{OVERVIEW OF MAJOR STEPS}
\subsection{CREATING A MUTATOR PROGRAM}
\subsection{SETTING UP THE APPLICATION PROGRAM(MUTATEE)}
\subsection{RUNNING THE MUTATOR}
\subsection{OPTIMIZING DYNINST PERFORMANCE}
\subsubsection{Optimizing Mutator Performance}
\subsubsection{Optimizing Mutatees Performance}

\appendix
\section{Complete Examples}\label{appdx:complete_examples}
In this section we show two complete examples: the programs from Section 3 and a complete Dyninst program, retee.\\

\subsection{INSTURMENTING A FUNCTION}

\lstinputlisting[language=c++]{test.cpp}

\subsection{BINARY ANALYSIS}
\subsection{INSTRUMENTING MEMORY ACCESS}
\subsection{RETEE}
\section{Running the Test Cases}
\section{Common pitfalss}

\printindex[terms]

\pagebreak

\bibliography{reference}
\bibliographystyle{ieeetr}


\end{document}

