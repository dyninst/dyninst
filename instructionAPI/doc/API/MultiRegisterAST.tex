\subsection{MultiRegister Class}
\label{sec:MultiRegister}

A \code{MultiMultiRegister} object represents a ordered collection of registers as a single operand
As a \code{MultiMultiRegister} is a \code{Expression}, it may contain the physical register\'s contents if
they are known.
 

\begin{apient}
typedef dyn\_detail::boost::shared\_ptr<MultiRegister> Ptr
\end{apient}
\apidesc{
A type definition for a reference-counted pointer to a \code{MultiRegister}.
}

\begin{apient}
MultiRegister (MachRegister r)
\end{apient}
\apidesc{
Construct a register using the provided register object \code{r}. The
\code{MachRegister} datatype is Dyninst's register representation and
should not be constructed manually. 
}

\begin{apient}
MultiRegisterAST(std::vector<RegisterAST::Ptr> _in);
\end{apient}


\begin{apient}
void getChildren (vector< InstructionAST::Ptr > & children) const
\end{apient}
\apidesc{
By definition, a \code{MultiRegister} object has no children.
Since a \code{MultiRegister} has no children, the \code{children} parameter is unchanged by this
method.
}

\begin{apient}
void getUses (set< InstructionAST::Ptr > & uses)
\end{apient}
\apidesc{
By definition, the use set of a \code{MultiRegister} object is itself.
This \code{MultiRegister} will be inserted into \code{uses}.
}

\begin{apient}
bool isUsed (InstructionAST::Ptr findMe) const
\end{apient}
\apidesc{
\code{isUsed} returns \code{true} if \code{findMe} is a \code{MultiRegister} that represents the same register as this \code{MultiRegister}, and \code{false} otherwise.
}

\begin{apient}
std::string format (formatStyle how = defaultStyle) const
\end{apient}
\apidesc{
The format method on a \code{MultiRegister} object returns the name associated with its ID.
}

\begin{apient}
MultiRegister makePC (Dyninst::Architecture arch) [static]
\end{apient}
\apidesc{
Utility function to get a \code{Register} object that represents the program counter.
\code{makePC} is provided to support platform-independent control flow analysis.
}

\begin{apient}
bool operator< (const MultiRegister & rhs) const
\end{apient}
\apidesc{
We define a partial ordering on registers by their register number so that they may be placed
into sets or other sorted containers.
}

\begin{apient}
MachRegister getID () const
\end{apient}
\apidesc{
The \code{getID} function returns underlying register represented by
this AST. 
}

\begin{apient}
MultiRegister::Ptr promote (const InstructionAST::Ptr reg) [static]
\end{apient}
\apidesc{
Utility function to hide aliasing complexity on platforms (IA-32) that allow addressing part or all of a register
}
